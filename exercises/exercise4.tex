\documentclass[./../main.tex]{subfiles}
\graphicspath{{img/}}
\begin{document}
	\begin{exercise}
		Cual sería la mínima energía necesaria para poder acelerar núcleos de \ch{Pb}. Aproxímalo como una partícula única y considera que el radio es de \SI{180e-12}{\m}. Utiliza la aproximación hecha en clase ¿tiene sentido? ¿A qué energía acelera los núcleos de \ch{Pb} el LHC?
	\end{exercise}
\end{document}
